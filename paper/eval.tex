\section{Applications and Evaluation}
\label{sec:eval}

We have implemented the interpolation algorithm in a modified version of the \dReal which is currently available at \url{https://github.com/dzufferey/dreal3/}\footnote{Once the implementation is stable, the interpolation will be merged into the released, main repository of \dReal}.
The interpolation hooks at the places where the proof is produced.
Since the proofs produced by \dReal can be very large,i.e., gigabytes, the interpolants are built and simplified on-the-fly.
The full proof is not kept.

\subsection{Flyspeck benchmarks}

The Flyspeck project~\cite{???} as build a mechanized proof of the Kepler conjecture.
Part of the proof requires showing the unsatisfiability of nonlinear formulas.
In previous work~\cite{???}, part of these examples where translated to the input format of \dReal.
We use this example as they provide a large number of tests and allow us to get quantitative information about our interpolation algorithm.

The ICP algorithm eagerly contract the domain by applying repeatedly \emph{all} the constraints. Therefore, it usually generate large proofs often involving all the constraints and all the variables.
Interpolation can extract more precise information from the proof.
In this test, we try to compare the size of the proof against the size of the interpolants.
Intuitively, an interpolant which is much smaller than the proof are more likely to be useful in practice.

From the flyspeck examples, we take the tests which contain conjunctions of constraints and associate A or B to each constraints with the restriction that each test must contains at least one constraint from each side.
We run \dReal, with a 10 minute timeout, and generate 562 interpolants.
Out of these, 92 are nontrivial.
In Figure~\ref{fig:flyspeck} we plot the number of inequalities in the nontrivial interpolants against the size of the proof.
For similar proofs, we see that the interpolants can be order of magnitude simpler than the proofs and other interpolants obtained by different partitions of the formula.
The trivial interpolants still bring information as they means that the only one side is part of the unsatisfiable core.
\todo{only 6 examples with non-trivial interpolants, and 19 counting the trivial ones}


\begin{figure}
\centering
\includegraphics[scale=0.6]{img/itpsize2.pdf}
\caption{
    Size of the interpolants (number of inequalities) compared to the size of the proof.
}
\label{fig:flyspeck}
\end{figure}


\subsection{Hybrid system verification}

... ToDo: rather about ODE than hybrid part ...
... what is hybrid ...
... bouncing ball: what to do when it hits the ground ...

...
In the ICP framework, the differential equations are just another kind of contractor and produce similar proofs.
Therefore, our algorithm compute interpolants for system with ODEs without modifications.
...


\subsection{Robotic design}

Most of the time, hybrid system verification is built around model-based design.
An expert produces a model of the system which is then verified.
However, it is also possible to extract models directly from the manufacturing designs.
As part of an ongoing project about co-design of both the software and hardware parts of robots~\cite{react}, we extract equations from robotic designs.
In the extracted model, each structural element is represented by a 3D vector for position and a unit quaternion for the orientation.
The dimension of the elements and the joints connecting them corresponds to equations that appropriately relate the position and orientation variables.
Active elements, such as motors, also have specific equations associated to them.

This approach has own sets of challenges as we get large system of very details equations.
To verify such systems, we need to simplify them.
Due to the presence of trigonometric functions we cannot use quantifier elimination for polynomial systems of equations~\cite{qepcad}.
However, we can use interpolation as an approximation of quantifier elimination.

\todo[inline]{explain how we get the equations}
% ... they are sparse/deterministic ...
% ... explain how we can generate a formula where the variables we don't want are located on one side only....
% ... picture from the tests ...

$\vec x$ are the control and design parameters,
$\vec y, \vec v$ are two copies the variables representing the state of each component of the robots,
$\vec z, \vec w$ are two copies of the part of the state needed to prove the property of interest, e.g., the position of the effector of an robotic manipulator.

$A$ is $𝓚(\vec x,\vec y,\vec z)$ and $B$ is $𝓚(\vec x,\vec v,\vec w) ∧ (\vec z-\vec w)² ≥ ε$ where $ε > δ$.

The interpolant $I(\vec x,\vec z)$ is an ε-approximation of $∃ \vec y.\,𝓚(\vec x,\vec y,\vec z)$.

