\section{Introduction}
\label{sec:intro}

Craig interpolation~\cite{MR0104564}
    what it is
        Given two formulas which conjunction is unsatisfiable, the interpolant is a formula over the common variable that captures the cause of the unsatisfiability.
    Applications 
        less expensive replacement for quantifier elimination
        heuristic to compute inductive invariant~\cite{DBLP:conf/cav/McMillan03,DBLP:conf/vmcai/McMillan07,DBLP:conf/sas/McMillan11}
        predicate discovery for abstraction refinement~\cite{DBLP:conf/cav/McMillan06}
        software verification~\cite{DBLP:conf/vmcai/AlbarghouthiGC12,DBLP:conf/cav/AlbarghouthiLGC12,DBLP:conf/esop/AlbarghouthiBCK15}
        fault-localisation~\cite{DBLP:conf/fm/ErmisSW12,DBLP:conf/vmcai/ChristESW13,DBLP:conf/sigsoft/SchafSW13}
    How it is done
        extracted from proof
    

Non-linear arithmetic over ℝ and δ-satisfiability
    what it is
        reasoning about continuous systems
    Applications 
        hybrid / embedded / cyber physical systems
    how it is done
        normal sat is undecidable
        δ-sat leverage on finite precision arithmetic what can be efficiently performed by computer

Flow
    get an unsatisfiability proof from a δ-decision procedure~\cite{DBLP:conf/synasc/GaoKC14}
    extract an interpolant from the proof
        decision procedure divides the space into small hypercubes and associating a constraint to each hypercube which shows its emptiness.
        The interpolation algorithm traverse the proof to construct an interpolant.
        To each leaf in the proof, we associate ⊤ or ⊥ depending on the source of the contradiction.
        The nodes corresponding to split are handled in a manner reminiscent of Pudl{\'a}k's algorithm~\cite{MR1472134}.
        Common variables are kept as branching points and $A$,$B$ local variables are eliminated.

A concrete example of what we can compute 
    the formula
    the proof
    the interpolant
    a picture

evaluation
    We have implemented this on top of \dReal
    and we tested it with examples from geometric theorem proving, robotic design, and hybrid system verification.  

% Outline of the paper
In Section~\ref{sec:prelim}, we review notions related to interpolation, nonlinear arithmetic over the Reals and δ-decision procedures.
In Section~\ref{sec:itp}, we introduce our interpolation algorithm.
In Section~\ref{sec:eval}, we present and evaluate our implementation.
We review the related work in Section~\ref{sec:related}.
Finally, we conclude and sketch future research direction in Section~\ref{sec:concl}.
