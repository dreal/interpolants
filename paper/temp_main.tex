\documentclass{llncs}

\usepackage{graphicx,epsfig,amssymb,amsmath,todonotes}
\usepackage{mathpartir}

\usepackage[utf8]{inputenc}
\usepackage{newunicodechar}
\newunicodechar{∃}{\ensuremath{\exists}}
\newunicodechar{∀}{\ensuremath{\forall}}
\newunicodechar{θ}{\ensuremath{\theta}}
\newunicodechar{τ}{\ensuremath{\tau}}
\newunicodechar{φ}{\ensuremath{\varphi}}
\newunicodechar{π}{\ensuremath{\pi}}
\newunicodechar{α}{\ensuremath{\alpha}}
\newunicodechar{β}{\ensuremath{\beta}}
\newunicodechar{γ}{\ensuremath{\gamma}}
\newunicodechar{δ}{\ensuremath{\delta}}
\newunicodechar{ε}{\ensuremath{\epsilon}}
\newunicodechar{λ}{\ensuremath{\lambda}}
\newunicodechar{ρ}{\ensuremath{\rho}}
\newunicodechar{σ}{\ensuremath{\sigma}}
\newunicodechar{ω}{\ensuremath{\omega}}
\newunicodechar{Γ}{\ensuremath{\Gamma}}
\newunicodechar{Φ}{\ensuremath{\Phi}}
\newunicodechar{Δ}{\ensuremath{\Delta}}
\newunicodechar{Σ}{\ensuremath{\Sigma}}
\newunicodechar{Π}{\ensuremath{\Pi}}
\newunicodechar{Θ}{\ensuremath{\Theta}}
\newunicodechar{⇒}{\ensuremath{\Rightarrow}}
\newunicodechar{⇐}{\ensuremath{\Leftarrow}}
\newunicodechar{⇔}{\ensuremath{\Leftrightarrow}}
\newunicodechar{→}{\ensuremath{\rightarrow}}
\newunicodechar{←}{\ensuremath{\leftarrow}}
\newunicodechar{↔}{\ensuremath{\leftrightarrow}}
\newunicodechar{¬}{\ensuremath{\neg}}
\newunicodechar{∧}{\ensuremath{\land}}
\newunicodechar{∨}{\ensuremath{\lor}}
\newunicodechar{≠}{\ensuremath{\neq}}
\newunicodechar{≡}{\ensuremath{\equiv}}
\newunicodechar{∼}{\ensuremath{\sim}}
\newunicodechar{≈}{\ensuremath{\approx}}
\newunicodechar{≥}{\ensuremath{\geq}}
\newunicodechar{≤}{\ensuremath{\leq}}
\newunicodechar{≫}{\ensuremath{\gg}}
\newunicodechar{≪}{\ensuremath{\ll}}
\newunicodechar{∅}{\ensuremath{\emptyset}}
\newunicodechar{⊆}{\ensuremath{\subseteq}}
\newunicodechar{⊂}{\ensuremath{\subset}}
\newunicodechar{∩}{\ensuremath{\cap}}
\newunicodechar{∪}{\ensuremath{\cup}}
\newunicodechar{∈}{\ensuremath{\in}}
\newunicodechar{⊤}{\ensuremath{\top}}
\newunicodechar{⊥}{\ensuremath{\bot}}
\newunicodechar{₀}{\ensuremath{_0}}
\newunicodechar{₁}{\ensuremath{_1}}
\newunicodechar{₂}{\ensuremath{_2}}
\newunicodechar{₃}{\ensuremath{_3}}
\newunicodechar{₄}{\ensuremath{_4}}
\newunicodechar{₅}{\ensuremath{_5}}
\newunicodechar{₆}{\ensuremath{_6}}
\newunicodechar{₇}{\ensuremath{_7}}
\newunicodechar{₈}{\ensuremath{_8}}
\newunicodechar{₉}{\ensuremath{_9}}
\newunicodechar{⁰}{\ensuremath{^0}}
\newunicodechar{¹}{\ensuremath{^1}}
\newunicodechar{²}{\ensuremath{^2}}
\newunicodechar{³}{\ensuremath{^3}}
\newunicodechar{⁴}{\ensuremath{^4}}
\newunicodechar{⁵}{\ensuremath{^5}}
\newunicodechar{⁶}{\ensuremath{^6}}
\newunicodechar{⁷}{\ensuremath{^7}}
\newunicodechar{⁸}{\ensuremath{^8}}
\newunicodechar{⁹}{\ensuremath{^9}}
\newunicodechar{𝔹}{\ensuremath{\mathbb{B}}}
\newunicodechar{ℝ}{\ensuremath{\mathbb{R}}}
\newunicodechar{ℕ}{\ensuremath{\mathbb{N}}}
\newunicodechar{ℂ}{\ensuremath{\mathbb{C}}}
\newunicodechar{ℚ}{\ensuremath{\mathbb{Q}}}
\newunicodechar{ℤ}{\ensuremath{\mathbb{Z}}}
\newunicodechar{✓}{\checkmark}
\newunicodechar{✗}{\ensuremath{\times}}
\newunicodechar{◊}{\ensuremath{\lozenge}}
\newunicodechar{□}{\ensuremath{\square}}


\newcommand{\lrf}{\mathcal{L}_{\mathbb{R}_{\mathcal{F}}}}
%\newtheorem{definition}{Definition}
%\newtheorem{notation}{Notation}
%\newtheorem{example}{Example}
%\newtheorem{proposition}{Proposition}
%\newtheorem{remark}{Remark}
%\newtheorem{theorem}{Theorem}

\begin{document}

\title{Interpolating Nonlinear Formulas over the Reals}
\author{Damien Zufferey \and Sicun Gao}
\institute{MIT}

\maketitle
\begin{abstract}
We develop algorithms for computing Craig interpolants for first-order formulas over real numbers with a wide range of nonlinear functions, including transcendental functions and differential equations. We transform proof traces from delta-complete decision procedures into interpolants that consist of Boolean combinations of linear constraints. The algorithms are guaranteed to find the interpolants between two formulas $A$ and $B$ whenever $A\wedge B$ is not delta-satisfiable. At the same time, by exploiting delta-perturbations one can parameterize the algorithm to find interpolants with different positions between $A$ and $B$. We show applications of the methods in geometric theorom proving, robotic design, and hybrid system verification.  
\end{abstract}

\section{Introduction}

Difficulty. 

A concrete example of what we can compute. 

Applications. 

\section{Preliminaries}

\paragraph{Craig interpolation}
Given $A$,$B$ such that $A ∧ B$ is unsatisfiable, an interpolant $I$ is a formula satisfying:
\begin{itemize}
\item $A ⇒ I$;
\item $B ∧ I ⇒ ⊥$;
\item $fv(I) ⊆ fv(A) ∩ fv(B)$ where $fv$ returns the free variables in a formula.
\end{itemize}

\paragraph{Proofs from $\delta$-complete decision procedures.}

The complete theory behind proof extraction from $\delta$-decision procedure is available in \cite{}.
Here, we described a simplified version.
The proof is divides the solution space until it can prove, using interval arithmetic, that each small piece of the solution space is empty.
The solver uses constraints propagation and pruning steps to limit the amount of branching required.
However, the proof it produces only need to consider two main rules: splitting and theory lemma.
The \textsc{Split} rules divides the solution space into two disjoint subspaces.
The theory lemmas (\textsc{ThLem}) are the leaves of the proof.
They occurs when the solver managed to prove the absence of solution in a given subspace.
The interval constraints propagation algorithms works on one constraint at the time. 
The \textsc{Weakening} rules extract those conjunct out of the main formula.

Each step of the proof has a set of variables $\vec x$ with a domain $\vec D$ and $F$ is a formula.
We use of vectors in the formulas, writing $\vec x ∈ \vec D$ to denote $\bigwedge_i x_i ∈ D_i$.
The domains are intervals, i.e., each $D_i$ has the form $[l_i,u_i]$ where $l_i$,$u_i$ are the lower and upper bounds for $x_i$.
Since we are looking at unsatisfiability proofs, each node implies $⊥$.

The root of the proof is has formula $A ∧ B$ and $D$ covers the entire domain,
the inner nodes are \textsc{Split}s,
and the proof's leaves are theory lemmas directly followed by a weakening.

\begin{mathpar}
\inferrule{ {} }{
  \vec x ∈ \vec D ∧ c ⇒ ⊥
}{\textsc{ThLem}}

\inferrule{
  D_i = [l;u) \\
  l < p < u \\\\
  x_i ∈ [l;p) ∧ \bigwedge_{j ≠ i} x_j ∈ D_j ∧ C ⇒ ⊥ \\
  x_i ∈ [p;u) ∧ \bigwedge_{j ≠ i} x_j ∈ D_j ∧ C ⇒ ⊥
}{
  \vec x∈\vec D ∧ C ⇒ ⊥
}{\textsc{Split}}

\inferrule{
  C = c ∧ \bigwedge_i c_i \\
  \vec x ∈ \vec D ∧ c ⇒ ⊥
}{
  \vec x ∈ \vec D ∧ C ⇒ ⊥
}{\textsc{Weakening}}
\end{mathpar}

\begin{example}

\end{example}



\section{Interpolants from Proofs}

\subsection{Interpolant Generation}

Introduce various parameters/templates for the interpolants. 
\begin{itemize}
	\item Robustness
	\item Boolean operations
	\item Degrees 
\end{itemize}

\begin{remark}[δ-interpolants]
The interpolation method that we propose uses a δ-decision procedure to build a craig interpolant.
The properties of the interpolant means that $A ∧ ¬I$ and $B ∧ I$ are both unsatisfiable.
However, they are not necessarily δ-unsatisfiable.

To obtain an interpolant such that both $A ∧ ¬I$ and $B ∧ I$ are δ-unsatisfiable, we can weaken both $A$ and $B$ by a factor δ.
However, $A$ and $B$ must be at least $3δ$-unsatisfiable to guarantee that the solver finds a proof of unsatisfiability.
\end{remark}

Algorithms for the generation from proof trees to disjunctions of linear constraints. 

Let $l$ be a labelling function that maps formula and variables to \textsc{a},\textsc{b}, or \textsc{ab}.
For each proof rule we associate an partial interpolant, written in square bracket on the right of the conclusion of the rules.

% need a labelling function
% ThLem: A → false 
%        B → true
% Split: A → I₁ ∨ I₂
%       AB → ite(x_i ≤ p, I₁, I₂) 
%        B → I₁ ∧ I₂
% Weakening: identify

\begin{mathpar}
\inferrule{ {} }{
  \vec x ∈ \vec D ∧ f ⇒ ⊥ \quad [l(f) ≠ \textsc{a}]
}{\textsc{ThLemI}}

\inferrule{
  D_i = [l;u) \\
  l < p < u \\\\
  x_i ∈ [l;p) ∧ \bigwedge_{j ≠ i} x_j ∈ D_j ∧ F ⇒ ⊥ \quad [I₁] \\
  x_i ∈ [p;u) ∧ \bigwedge_{j ≠ i} x_j ∈ D_j ∧ F ⇒ ⊥ \quad [I₂] 
}{
  \vec x ∈ \vec D ∧ F ⇒ ⊥ \quad
  \left[ \substack{ I₁ ∨ I₂     \qquad \quad ~~  \text{if} ~ l(x_i) = \textsc{a} \\
                    ite(x_i < p, I₁, I₂) ~~ \text{if} ~ l(x_i) = \textsc{ab}\\
                    I₁ ∧ I₂     \qquad \quad ~~  \text{if} ~ l(x_i) = \textsc{b}}\right]
}{\textsc{SplitI}}

\inferrule{
  F = f ∧ \bigwedge_k F_k \\
  \vec x ∈ \vec D ∧ f ⇒ ⊥ \quad [I]
}{
  \vec x ∈ \vec D ∧ F ⇒ ⊥ \quad [I]
}{\textsc{WeakeningI}}
\end{mathpar}

where $ite(x,y,z)$ is a shorthand for $(x ∧ y)∨(¬x ∧ z)$

\todo[inline]{find a better way to format that.}

Intuitively, a proof of unsatisfiability is a tiling of the solution space where each tile is associated with a conjunct $f$ from $A ∧ B$.
$f$ is a witness that shows the absence of solution in a given tile.
The interpolation rules traverse the rules and selects which tiles belong to the interpolant $I$.

At the leaf level (rule \textsc{ThLemI}), the tile is in $I$ if $f$ is not part of $A$, i.e., the contradiction originates from $B$.
If $f$ is in both $A$ and $B$ then it can be consideres as either part of $A$ or $B$.
Both cases leads to a correct interpolant.
The \textsc{WeakeningI} rule does not influence the interpolant, it is only required to pick $f$ from $A ∧ B$.

The \textsc{SplitI} is the most interesting rule.
Splitting the domains essentially defined the bounds of the subsequent tiles.
Let $x$ be the variable whose domain is split at value $p$ and $I₁$, $I₂$ be the two interpolants for the case when $x < p$ and $x ≥ p$.
If $x$ occurs in $A$ but not $B$, then $x$ cannot occur in $I$.
Since $x$ is in $A$ then we know that $A$ implies $x < p ⇒ I₁$ and $x ≥ p ⇒ I₂$.
Eliminating $x$ give $I = I₁ ∨ I₂$.
A similar reasoning is applicable when $x$ occurs in $B$ but not $A$ and gives $I = I₁ ∧ I₂$.
When $x$ occurs in both $A$ and $B$ then $x$ is kept in $I$ and acts as a selector for the values of $x$ smaller than $p$ $I₁$ is selected, otherwise $I₂$ applies.

\begin{theorem}
The proof rules are correct. 
\end{theorem}
\section{Proofs}

TODO better formating ...

Correctness of the interpolation rules:

Proof by induction.

\paragraph{Base case: \textsc{ThLemI} rule}
if $l(f)=A$ and $I=⊥$ then
\begin{itemize}
\item $A ⇒ I$: if $l(f)=A$ the theroy lemma means that $A⇔⊥$, thus, $⊥⇒⊥$
\item $B ∧ I ⇒ ⊥$: trivially $⊥⇒⊥$
\item $fv(I) ⊆ fv(A) ∩ fv(B)$: trivial
\end{itemize}

else we have $l(f)=B$ and $I=\top$, so
\begin{itemize}
\item $A ⇒ I$: trivially $\top ⇒ \top$
\item $B ∧ I ⇒ ⊥$:  if $l(f)=B$ the theroy lemma means that $A⇔⊥$, thus, $⊥⇒⊥$
\item $fv(I) ⊆ fv(A) ∩ fv(B)$: trivial
\end{itemize}

\paragraph{Induction step 1: \textsc{WeakeningI} rule}
\begin{itemize}
\item $A ⇒ I$: by induction hypothesis
\item $B ∧ I ⇒ ⊥$: by induction hypothesis
\item $fv(I) ⊆ fv(A) ∩ fv(B)$: by induction hypothesis
\end{itemize}

\paragraph{Induction step 2: \textsc{SplitI} rule}
The trick is how to apply the induction hypothesis.
We have to consider 3 cases:
\begin{enumerate}
\item $x ∈ fv(A) ∧ x \notin fv(B)$ and $I = I₁ ∨ I₂$
\item $x ∈ fv(A) ∧ x ∈ fv(B)$ and $I = ite(x < p, I₁, I₂)$
\item $x \notin fv(A) ∧ x ∈ fv(B)$ and $I = I₁ ∧ I₂$
\end{enumerate}

Let's divide the domain $D$ into $D_A$ and $D_B$ where each contains the range of the corresponding variables.
Variables that are in both side have their domain in both $D_A$ and $D_B$. 
if $x\in fv(A)$ then the induction step has $D_{A1} = D_A ∧ x < p$ and $D_{A2} = D_A ∧ x ≥ p$ and $D_B$ is unchanged.

In the first case, the induction hypothesis is
\begin{eqnarray*}
& A ∧ x_i < p ⇒ I₁ \\
& A ∧ x_i ≥ p ⇒ I₂ \\
& B ∧ I₁ ⇒ ⊥ \\
& B ∧ I₂ ⇒ ⊥
\end{eqnarray*}
which simplifies to:
\begin{eqnarray*}
& A ⇒ I₁ ∨ I₂ \\
& B ∧ (I₁ ∨ I₂) ⇒ ⊥
\end{eqnarray*}

In the second case, the induction hypothesis is
\begin{eqnarray*}
& A ∧ x_i < p ⇒ I₁ \\
& A ∧ x_i ≥ p ⇒ I₂ \\
& B ∧ x_i < p ∧ I₁ ⇒ ⊥ \\
& B ∧ x_i ≥ p ∧ I₂ ⇒ ⊥
\end{eqnarray*}

The first two equations gives
\[
    ¬ A ∨ (¬(x < p) ∨ I₁) ∧ (¬(x ≥ p) ∨ I₂)
\]
which is the same as $¬ A ∨ ite(x < p, I₁, I₂)$
and, therefore, $A ⇒ ite(x < p, I₁, I₂)$.

The last two equations gives
\[
    ¬ B ∨ ((¬(x < p) ∨ ¬ I₁) ∧ (¬(x ≥ p) ∨ ¬ I₂))
\]
using distributivity law we get
\[
    ¬ B ∨ (x ≥ p ∧ ¬ I₂) ∨ (x < p ∧ ¬ I₁) ∨ (¬ I₁ ∧ ¬ I₂)
\]
because $(¬ I₁ ∧ ¬ I₂) ⇒  (x ≥ p ∧ ¬ I₂) ∨ (x < p ∧ ¬ I₁)$
this simplifies to
\[
 ¬ B ∨ (x ≥ p ∧ ¬ I₂) ∨ (x < p ∧ ¬ I₁)
\]
which is the same as: $¬ B ∨ ¬ ite(x < p, I₁, I₂)$
and, therefore, $B ∧ ite(x < p, I₁, I₂) ⇒ ⊥$.

In the third case: the induction hypothesis is
\begin{eqnarray*}
& A ⇒ I₁ \\
& A ⇒ I₂ \\
& B ∧ x_i < p ∧ I₁ ⇒ ⊥ \\
& B ∧ x_i ≥ p ∧ I₂ ⇒ ⊥
\end{eqnarray*}
which simplifies to:
\begin{eqnarray*}
& A ⇒ I₁ ∧ I₂ \\
& B ∧ (I₁ ∧ I₂) ⇒ ⊥
\end{eqnarray*}


%\section{Smoothing}

%Algorithms for smoothing DL-interpolants to user specified classes of nonlinear interpolants. 


\subsection{Handling Differential Equations}

\section{Parameterized Interpolant Generation}

\subsection{Distances}

\subsection{Nonlinear interpolants}

Parameterization 


\section{Examples and Experiments}


%\appendix


\end{document}

