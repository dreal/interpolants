\section{Interpolants in Nonlinear Theories}
\label{sec:itp}

Introduce various parameters/templates for the interpolants. 
\begin{itemize}
	\item Robustness
	\item Boolean operations
	\item Degrees 
\end{itemize}

\begin{remark}[δ-interpolants]
The interpolation method that we propose uses a δ-decision procedure to build a craig interpolant.
The properties of the interpolant means that $A ∧ ¬I$ and $B ∧ I$ are both unsatisfiable.
However, they are not necessarily δ-unsatisfiable.

To obtain an interpolant such that both $A ∧ ¬I$ and $B ∧ I$ are δ-unsatisfiable, we can weaken both $A$ and $B$ by a factor δ.
However, $A$ and $B$ must be at least $3δ$-unsatisfiable to guarantee that the solver finds a proof of unsatisfiability.
\end{remark}


\subsection{Disjunctive Linear Interpolants}

Algorithms for the generation from proof trees to disjunctions of linear constraints. 

Let $l$ be a labelling function that maps formula and variables to \textsc{a},\textsc{b}, or \textsc{ab}.
For each proof rule we associate an partial interpolant, written in square bracket on the right of the conclusion of the rules.

% need a labelling function
% ThLem: A → false 
%        B → true
% Split: A → I₁ ∨ I₂
%       AB → ite(x_i ≤ p, I₁, I₂) 
%        B → I₁ ∧ I₂
% Weakening: identify

\begin{mathpar}
\inferrule{ {} }{
  \vec x ∈ \vec D ∧ f ⇒ ⊥ \quad [l(f) ≠ \textsc{a}]
}{\textsc{ThLemI}}

\inferrule{
  D_i = [l;u) \\
  l < p < u \\\\
  x_i ∈ [l;p) ∧ \bigwedge_{j ≠ i} x_j ∈ D_j ∧ F ⇒ ⊥ \quad [I₁] \\
  x_i ∈ [p;u) ∧ \bigwedge_{j ≠ i} x_j ∈ D_j ∧ F ⇒ ⊥ \quad [I₂] 
}{
  \vec x ∈ \vec D ∧ F ⇒ ⊥ \quad
  \left[ \substack{ I₁ ∨ I₂     \qquad \quad ~~  \text{if} ~ l(x_i) = \textsc{a} \\
                    ite(x_i < p, I₁, I₂) ~~ \text{if} ~ l(x_i) = \textsc{ab}\\
                    I₁ ∧ I₂     \qquad \quad ~~  \text{if} ~ l(x_i) = \textsc{b}}\right]
}{\textsc{SplitI}}

\inferrule{
  F = f ∧ \bigwedge_k F_k \\
  \vec x ∈ \vec D ∧ f ⇒ ⊥ \quad [I]
}{
  \vec x ∈ \vec D ∧ F ⇒ ⊥ \quad [I]
}{\textsc{WeakeningI}}
\end{mathpar}

where $ite(x,y,z)$ is a shorthand for $(x ∧ y)∨(¬x ∧ z)$

\todo[inline]{find a better way to format that.}

Intuitively, a proof of unsatisfiability is a tiling of the solution space where each tile is associated with a conjunct $f$ from $A ∧ B$.
$f$ is a witness that shows the absence of solution in a given tile.
The interpolation rules traverse the rules and selects which tiles belong to the interpolant $I$.

At the leaf level (rule \textsc{ThLemI}), the tile is in $I$ if $f$ is not part of $A$, i.e., the contradiction originates from $B$.
If $f$ is in both $A$ and $B$ then it can be consideres as either part of $A$ or $B$.
Both cases leads to a correct interpolant.
The \textsc{WeakeningI} rule does not influence the interpolant, it is only required to pick $f$ from $A ∧ B$.

The \textsc{SplitI} is the most interesting rule.
Splitting the domains essentially defined the bounds of the subsequent tiles.
Let $x$ be the variable whose domain is split at value $p$ and $I₁$, $I₂$ be the two interpolants for the case when $x < p$ and $x ≥ p$.
If $x$ occurs in $A$ but not $B$, then $x$ cannot occur in $I$.
Since $x$ is in $A$ then we know that $A$ implies $x < p ⇒ I₁$ and $x ≥ p ⇒ I₂$.
Eliminating $x$ give $I = I₁ ∨ I₂$.
A similar reasoning is applicable when $x$ occurs in $B$ but not $A$ and gives $I = I₁ ∧ I₂$.
When $x$ occurs in both $A$ and $B$ then $x$ is kept in $I$ and acts as a selector for the values of $x$ smaller than $p$ $I₁$ is selected, otherwise $I₂$ applies.

\todo[inline]{the main theorem}

\subsection{Extensions}

\paragraph{Boolean structure}
... combines with traditional boolean interpolation schemes as presented by Yorsh and Musuvathi~\cite{DBLP:conf/cade/YorshM05} ...

\paragraph{Interpolant strength}
... also possible to apply methods such as D'Silva et.al.~\cite{DBLP:conf/vmcai/DSilvaKPW10} ...

