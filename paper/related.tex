\section{Related Work}
\label{sec:related}

Propositional interpolation was studied by Pudl{\'a}k~\cite{MR1472134}.
Our algorithm is very similar, however, it applies to a very different theory.

Craig interpolation for theories of the reals or integers have focused on the 
linear fragment with LA(ℚ)~\cite{DBLP:conf/tacas/McMillan04,DBLP:conf/vmcai/RybalchenkoS07} and LA(ℤ)~\cite{DBLP:conf/cade/BrilloutKRW10,DBLP:conf/tacas/GriggioLS11}.
Our method currently handles non-linear theories over ℝ.
We plan to expand our work to a richer proof system that also handles ℤ.

Existing tools to compute interpolation such as
\smtinterpol~\cite{DBLP:conf/spin/ChristHN12},
\princess~\cite{DBLP:conf/cade/BrilloutKRW10},
\zthree~\cite{DBLP:conf/fmcad/McMillan11},
\mathsat~\cite{mathsat5} focus on linear arithmetic.
We are the first to provide interpolant in nonlinear theories.

\todo[inline]{some more, about hybrid stuff, related notions like QE, ...}
