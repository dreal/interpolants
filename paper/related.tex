\section{Related Work and Future Work}
\label{sec:related}

Propositional interpolation was studied by Pudl{\'a}k~\cite{MR1472134}.
Our algorithm is very similar, however, it applies to a very different theory.

Craig interpolation for theories of the reals or integers have focused on the 
linear fragment with LA(ℚ)~\cite{DBLP:conf/tacas/McMillan04,DBLP:conf/vmcai/RybalchenkoS07} and LA(ℤ)~\cite{DBLP:conf/cade/BrilloutKRW10,DBLP:conf/tacas/GriggioLS11}.
Our method currently handles nonlinear theories over ℝ.

Existing tools to compute interpolation such as
\mathsat~\cite{mathsat5},
\princess~\cite{DBLP:conf/cade/BrilloutKRW10},
\smtinterpol~\cite{DBLP:conf/spin/ChristHN12}, and
\zthree~\cite{DBLP:conf/fmcad/McMillan11}
focus on linear arithmetic.
We are the first to provide interpolant in nonlinear theories.

In the future, we plan to expand our work to a richer proof systems.
The ICP loop produces proof based on interval pruning which results in large, ``squarish'' interpolants.
Using more general proof systems, e.g. cutting planes, we will be able to get smaller, smoother interpolants.
CDCL-style reasoning for richer theories, e.g., LA(ℚ)~\cite{DBLP:conf/cav/McMillanKS09} and polynomial~\cite{DBLP:conf/cade/JovanovicM12}, is a likely basis for such extensions.
Furthermore, we are interested in investigating the link between classical interpolation and Craig interpolation over the reals.
Using methods like spline interpolation and radial basis functions, it maybe possible to build smoother interpolants.
%We also to extend the our rules to compute interpolants mixed proofs with both integer and real variables.
