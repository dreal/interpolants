\documentclass{acm_proc_article-sp}

\usepackage{graphicx,epsfig,amssymb,amsmath}
\usepackage{mathpartir}

\usepackage[utf8]{inputenc}
\usepackage{newunicodechar}
\newunicodechar{∃}{\ensuremath{\exists}}
\newunicodechar{∀}{\ensuremath{\forall}}
\newunicodechar{θ}{\ensuremath{\theta}}
\newunicodechar{τ}{\ensuremath{\tau}}
\newunicodechar{φ}{\ensuremath{\varphi}}
\newunicodechar{π}{\ensuremath{\pi}}
\newunicodechar{α}{\ensuremath{\alpha}}
\newunicodechar{β}{\ensuremath{\beta}}
\newunicodechar{γ}{\ensuremath{\gamma}}
\newunicodechar{δ}{\ensuremath{\delta}}
\newunicodechar{ε}{\ensuremath{\epsilon}}
\newunicodechar{λ}{\ensuremath{\lambda}}
\newunicodechar{ρ}{\ensuremath{\rho}}
\newunicodechar{σ}{\ensuremath{\sigma}}
\newunicodechar{ω}{\ensuremath{\omega}}
\newunicodechar{Γ}{\ensuremath{\Gamma}}
\newunicodechar{Φ}{\ensuremath{\Phi}}
\newunicodechar{Δ}{\ensuremath{\Delta}}
\newunicodechar{Σ}{\ensuremath{\Sigma}}
\newunicodechar{Π}{\ensuremath{\Pi}}
\newunicodechar{Θ}{\ensuremath{\Theta}}
\newunicodechar{⇒}{\ensuremath{\Rightarrow}}
\newunicodechar{⇐}{\ensuremath{\Leftarrow}}
\newunicodechar{⇔}{\ensuremath{\Leftrightarrow}}
\newunicodechar{→}{\ensuremath{\rightarrow}}
\newunicodechar{←}{\ensuremath{\leftarrow}}
\newunicodechar{↔}{\ensuremath{\leftrightarrow}}
\newunicodechar{¬}{\ensuremath{\neg}}
\newunicodechar{∧}{\ensuremath{\land}}
\newunicodechar{∨}{\ensuremath{\lor}}
\newunicodechar{≠}{\ensuremath{\neq}}
\newunicodechar{≡}{\ensuremath{\equiv}}
\newunicodechar{∼}{\ensuremath{\sim}}
\newunicodechar{≈}{\ensuremath{\approx}}
\newunicodechar{≥}{\ensuremath{\geq}}
\newunicodechar{≤}{\ensuremath{\leq}}
\newunicodechar{≫}{\ensuremath{\gg}}
\newunicodechar{≪}{\ensuremath{\ll}}
\newunicodechar{∅}{\ensuremath{\emptyset}}
\newunicodechar{⊆}{\ensuremath{\subseteq}}
\newunicodechar{⊂}{\ensuremath{\subset}}
\newunicodechar{∩}{\ensuremath{\cap}}
\newunicodechar{∪}{\ensuremath{\cup}}
\newunicodechar{∈}{\ensuremath{\in}}
\newunicodechar{⊤}{\ensuremath{\top}}
\newunicodechar{⊥}{\ensuremath{\bot}}
\newunicodechar{₀}{\ensuremath{_0}}
\newunicodechar{₁}{\ensuremath{_1}}
\newunicodechar{₂}{\ensuremath{_2}}
\newunicodechar{₃}{\ensuremath{_3}}
\newunicodechar{₄}{\ensuremath{_4}}
\newunicodechar{₅}{\ensuremath{_5}}
\newunicodechar{₆}{\ensuremath{_6}}
\newunicodechar{₇}{\ensuremath{_7}}
\newunicodechar{₈}{\ensuremath{_8}}
\newunicodechar{₉}{\ensuremath{_9}}
\newunicodechar{⁰}{\ensuremath{^0}}
\newunicodechar{¹}{\ensuremath{^1}}
\newunicodechar{²}{\ensuremath{^2}}
\newunicodechar{³}{\ensuremath{^3}}
\newunicodechar{⁴}{\ensuremath{^4}}
\newunicodechar{⁵}{\ensuremath{^5}}
\newunicodechar{⁶}{\ensuremath{^6}}
\newunicodechar{⁷}{\ensuremath{^7}}
\newunicodechar{⁸}{\ensuremath{^8}}
\newunicodechar{⁹}{\ensuremath{^9}}
\newunicodechar{𝔹}{\ensuremath{\mathbb{B}}}
\newunicodechar{ℝ}{\ensuremath{\mathbb{R}}}
\newunicodechar{ℕ}{\ensuremath{\mathbb{N}}}
\newunicodechar{ℂ}{\ensuremath{\mathbb{C}}}
\newunicodechar{ℚ}{\ensuremath{\mathbb{Q}}}
\newunicodechar{ℤ}{\ensuremath{\mathbb{Z}}}
\newunicodechar{✓}{\checkmark}
\newunicodechar{✗}{\ensuremath{\times}}
\newunicodechar{◊}{\ensuremath{\lozenge}}
\newunicodechar{□}{\ensuremath{\square}}


\newcommand{\lrf}{\mathcal{L}_{\mathbb{R}_{\mathcal{F}}}}
\newtheorem{definition}{Definition}
\newtheorem{notation}{Notation}
\newtheorem{example}{Example}
\newtheorem{proposition}{Proposition}
\newtheorem{remark}{Remark}
\newtheorem{theorem}{Theorem}

\begin{document}

\title{Interpolants in Nonlinear Theories over the Reals}
\numberofauthors{2}
\author{
\alignauthor
Damien Zufferey\\
\alignauthor
Sicun Gao
}
\maketitle
\begin{abstract}
We develop methods for obtaining interpolants for first-order formulas over the reals with a wide range of nonlinear functions. 
\end{abstract}

\section{Introduction}

Difficulty. 

A concrete example of what we can compute. 

Applications. 

\section{Preliminaries}

\paragraph{Craig interpolation}
Given $A$,$B$ such that $A ∧ B$ is unsatisfiable, an interpolant $I$ is a formula satisfying:
\begin{itemize}
\item $A ⇒ I$;
\item $I ∧ B$ is unsatisfiable;
\item $I$ contains only variables common to $A$ and $B$.
\end{itemize}

\paragraph{δ-satifiability}

\paragraph{δ-unsatisfiability proof}

The complete theory behind proof extraction from $\delta$-decision procedure is available in \cite{}.
Here, we described a simplified version.
The proof is divides the solution space until it can prove, using interval arithmetic, that each small piece of the solution space is empty.
The solver uses constraints propagation and pruning steps to limit the amount of branching required.
However, the proof it produces only need to consider two main rules: splitting and theory lemma.
The \textsc{Split} rules divides the solution space into two disjoint subspaces.
The theory lemmas (\textsc{ThLem}) are the leaves of the proof.
They occurs when the solver managed to prove the absence of solution in a given subspace.
The interval constraints propagation algorithms works on one constraint at the time. 
The \textsc{Weakening} rules extract those conjunct out of the main formula.

Each step of the proof has a set of variables $\vec x$ with a domain $\vec D$ and $F$ is a formula.
We use of vectors in the formulas, writing $\vec x ∈ \vec D$ to denote $\bigwedge_i x_i ∈ D_i$.
The domains are intervals, i.e., each $D_i$ has the form $[l_i,u_i]$ where $l_i$,$u_i$ are the lower and upper bounds for $x_i$.
Since we are looking at unsatisfiability proofs, each node implies $⊥$.

The root of the proof is has formula $A ∧ B$ and $D$ covers the entire domain,
the inner nodes are \textsc{Split}s,
and the proof's leaves are theory lemmas directly followed by a weakening.

\begin{mathpar}
\inferrule{ {} }{
  \vec x ∈ \vec D ∧ f → ⊥
}{\textsc{ThLem}}

\inferrule{
  D_i = [l;u] \\
  l < p < u \\\\
  x_i ∈ [l;p] ∧ \bigwedge_{j ≠ i} x_j ∈ D_j ∧ F → ⊥ \\
  x_i ∈ [p;u] ∧ \bigwedge_{j ≠ i} x_j ∈ D_j ∧ F → ⊥
}{
  \vec x∈\vec D ∧ F → ⊥
}{\textsc{Split}}

\inferrule{
  F = f ∧ \bigwedge_k F_k \\
  \vec x ∈ \vec D ∧ f → ⊥
}{
  \vec x ∈ \vec D ∧ F → ⊥
}{\textsc{Weakening}}
\end{mathpar}
TODO can we avoid weakening/strengthening ?

\section{Interpolants in Nonlinear Theories}

Introduce various parameters/templates for the interpolants. 
\begin{itemize}
	\item Robustness
	\item Boolean operations
	\item Degrees 
\end{itemize}

Is there something like δ-interpolants ?


\section{Disjunctive Linear Interpolants}

Algorithms for the generation from proof trees to disjunctions of linear constraints. 

Let $l$ be a labelling function that maps formula and variables to \textsc{a},\textsc{b}, or \textsc{ab}.
For each proof rule we associate an partial interpolant, written in square bracket on the right of the conclusion of the rules.

% need a labelling function
% ThLem: A → false 
%        B → true
% Split: A → I_1 ∨ I_2
%       AB → ite(x_i ≤ p, I_1, I_2) 
%        B → I_1 ∧ I_2
% Weakening: identify

\begin{mathpar}
\inferrule{ {} }{
  \vec x ∈ \vec D ∧ f → ⊥ \quad [l(f) ≠ \textsc{a}]
}{\textsc{ThLemI}}

\inferrule{
  D_i = [l;u] \\
  l < p < u \\\\
  x_i ∈ [l;p] ∧ \bigwedge_{j ≠ i} x_j ∈ D_j ∧ F → ⊥ \quad [I_1] \\
  x_i ∈ [p;u] ∧ \bigwedge_{j ≠ i} x_j ∈ D_j ∧ F → ⊥ \quad [I_2] \\
}{
  \vec x ∈ \vec D ∧ F → ⊥ \quad
  \left[ \substack{ I_1 ∨ I_2     \qquad \quad ~~  \text{if} ~ l(x_i) = \textsc{a} \\
                    ite(x_i ≤ p, I_1, I_2) ~~ \text{if} ~ l(x_i) = \textsc{ab}\\
                    I_1 ∧ I_2     \qquad \quad ~~  \text{if} ~ l(x_i) = \textsc{b}}\right]
}{\textsc{SplitI}}

\inferrule{
  F = f ∧ \bigwedge_k F_k \\
  \vec x ∈ \vec D ∧ f → ⊥ \quad [I]
}{
  \vec x ∈ \vec D ∧ F → ⊥ \quad [I]
}{\textsc{WeakeningI}}
\end{mathpar}

where $ite(x,y,z)$ is a shorthand for $(x ∧ y)∨(¬x ∧ z)$

TODO find a better way to format that.

TODO explain (and prove) that this is correct.

\section{Smoothing}

Algorithms for smoothing DL-interpolants to user specified classes of nonlinear interpolants. 


\section{Examples and Experiments}




\end{document}

