\section{Preliminaries}
\label{sec:prelim}

\paragraph{Craig interpolation}
Given $A$,$B$ such that $A ∧ B$ is unsatisfiable, an interpolant $I$ is a formula satisfying:
\begin{itemize}
\item $A ⇒ I$;
\item $B ∧ I ⇒ ⊥$;
\item $fv(I) ⊆ fv(A) ∩ fv(B)$ where $fv$ returns the free variables in a formula.
\end{itemize}

\paragraph{Proofs from $\delta$-complete decision procedures.}

The complete theory behind proof extraction from $\delta$-decision procedure is available in \cite{}.
Here, we described a simplified version.
The proof is divides the solution space until it can prove, using interval arithmetic, that each small piece of the solution space is empty.
The solver uses constraints propagation and pruning steps to limit the amount of branching required.
However, the proof it produces only need to consider two main rules: splitting and theory lemma.
The \textsc{Split} rules divides the solution space into two disjoint subspaces.
The theory lemmas (\textsc{ThLem}) are the leaves of the proof.
They occurs when the solver managed to prove the absence of solution in a given subspace.
The interval constraints propagation algorithms works on one constraint at the time. 
The \textsc{Weakening} rules extract those conjunct out of the main formula.

Each step of the proof has a set of variables $\vec x$ with a domain $\vec D$ and $F$ is a formula.
We use of vectors in the formulas, writing $\vec x ∈ \vec D$ to denote $\bigwedge_i x_i ∈ D_i$.
The domains are intervals, i.e., each $D_i$ has the form $[l_i,u_i]$ where $l_i$,$u_i$ are the lower and upper bounds for $x_i$.
Since we are looking at unsatisfiability proofs, each node implies $⊥$.

The root of the proof is has formula $A ∧ B$ and $D$ covers the entire domain,
the inner nodes are \textsc{Split}s,
and the proof's leaves are theory lemmas directly followed by a weakening.

\begin{mathpar}
\inferrule{ {} }{
  \vec x ∈ \vec D ∧ f ⇒ ⊥
}{\textsc{ThLem}}

\inferrule{
  D_i = [l;u) \\
  l < p < u \\\\
  x_i ∈ [l;p) ∧ \bigwedge_{j ≠ i} x_j ∈ D_j ∧ F ⇒ ⊥ \\
  x_i ∈ [p;u) ∧ \bigwedge_{j ≠ i} x_j ∈ D_j ∧ F ⇒ ⊥
}{
  \vec x∈\vec D ∧ F ⇒ ⊥
}{\textsc{Split}}

\inferrule{
  F = f ∧ \bigwedge_k F_k \\
  \vec x ∈ \vec D ∧ f ⇒ ⊥
}{
  \vec x ∈ \vec D ∧ F ⇒ ⊥
}{\textsc{Weakening}}
\end{mathpar}
\todo[inline]{can we avoid weakening/strengthening?}

