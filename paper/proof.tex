\section{Proofs}

TODO better formating ...

Correctness of the interpolation rules:

Proof by induction.

\paragraph{Base case: \textsc{ThLemI} rule}
if $l(f)=A$ and $I=⊥$ then
\begin{itemize}
\item $A ⇒ I$: if $l(f)=A$ the theroy lemma means that $A⇔⊥$, thus, $⊥⇒⊥$
\item $B ∧ I ⇒ ⊥$: trivially $⊥⇒⊥$
\item $fv(I) ⊆ fv(A) ∩ fv(B)$: trivial
\end{itemize}

else we have $l(f)=B$ and $I=\top$, so
\begin{itemize}
\item $A ⇒ I$: trivially $\top ⇒ \top$
\item $B ∧ I ⇒ ⊥$:  if $l(f)=B$ the theroy lemma means that $A⇔⊥$, thus, $⊥⇒⊥$
\item $fv(I) ⊆ fv(A) ∩ fv(B)$: trivial
\end{itemize}

\paragraph{Induction step 1: \textsc{WeakeningI} rule}
\begin{itemize}
\item $A ⇒ I$: by induction hypothesis
\item $B ∧ I ⇒ ⊥$: by induction hypothesis
\item $fv(I) ⊆ fv(A) ∩ fv(B)$: by induction hypothesis
\end{itemize}

\paragraph{Induction step 2: \textsc{SplitI} rule}
The trick is how to apply the induction hypothesis.
We have to consider 3 cases:
\begin{enumerate}
\item $x ∈ fv(A) ∧ x \notin fv(B)$ and $I = I₁ ∨ I₂$
\item $x ∈ fv(A) ∧ x ∈ fv(B)$ and $I = ite(x < p, I₁, I₂)$
\item $x \notin fv(A) ∧ x ∈ fv(B)$ and $I = I₁ ∧ I₂$
\end{enumerate}

Let's divide the domain $D$ into $D_A$ and $D_B$ where each contains the range of the corresponding variables.
Variables that are in both side have their domain in both $D_A$ and $D_B$. 
if $x\in fv(A)$ then the induction step has $D_{A1} = D_A ∧ x < p$ and $D_{A2} = D_A ∧ x ≥ p$ and $D_B$ is unchanged.

In the first case, the induction hypothesis is
\begin{eqnarray*}
& A ∧ x_i < p ⇒ I₁ \\
& A ∧ x_i ≥ p ⇒ I₂ \\
& B ∧ I₁ ⇒ ⊥ \\
& B ∧ I₂ ⇒ ⊥
\end{eqnarray*}
which simplifies to:
\begin{eqnarray*}
& A ⇒ I₁ ∨ I₂ \\
& B ∧ (I₁ ∨ I₂) ⇒ ⊥
\end{eqnarray*}

In the second case, the induction hypothesis is
\begin{eqnarray*}
& A ∧ x_i < p ⇒ I₁ \\
& A ∧ x_i ≥ p ⇒ I₂ \\
& B ∧ x_i < p ∧ I₁ ⇒ ⊥ \\
& B ∧ x_i ≥ p ∧ I₂ ⇒ ⊥
\end{eqnarray*}

The first two equations gives
\[
    ¬ A ∨ (¬(x < p) ∨ I₁) ∧ (¬(x ≥ p) ∨ I₂)
\]
which is the same as $¬ A ∨ ite(x < p, I₁, I₂)$
and, therefore, $A ⇒ ite(x < p, I₁, I₂)$.

The last two equations gives
\[
    ¬ B ∨ ((¬(x < p) ∨ ¬ I₁) ∧ (¬(x ≥ p) ∨ ¬ I₂))
\]
using distributivity law we get
\[
    ¬ B ∨ (x ≥ p ∧ ¬ I₂) ∨ (x < p ∧ ¬ I₁) ∨ (¬ I₁ ∧ ¬ I₂)
\]
because $(¬ I₁ ∧ ¬ I₂) ⇒  (x ≥ p ∧ ¬ I₂) ∨ (x < p ∧ ¬ I₁)$
this simplifies to
\[
 ¬ B ∨ (x ≥ p ∧ ¬ I₂) ∨ (x < p ∧ ¬ I₁)
\]
which is the same as: $¬ B ∨ ¬ ite(x < p, I₁, I₂)$
and, therefore, $B ∧ ite(x < p, I₁, I₂) ⇒ ⊥$.

In the third case: the induction hypothesis is
\begin{eqnarray*}
& A ⇒ I₁ \\
& A ⇒ I₂ \\
& B ∧ x_i < p ∧ I₁ ⇒ ⊥ \\
& B ∧ x_i ≥ p ∧ I₂ ⇒ ⊥
\end{eqnarray*}
which simplifies to:
\begin{eqnarray*}
& A ⇒ I₁ ∧ I₂ \\
& B ∧ (I₁ ∧ I₂) ⇒ ⊥
\end{eqnarray*}

